\begin{center}

\textbf{\thesistitle}\\

(Order No.\ \ \ \ \ \ \ \ \ \ \ \ \ )

\textbf{ALEXANDER KITT}\\
Boston University Graduate School of Arts and Sciences, 2014\\
\begin{singlespace}
	Major Professor: Bennett B. Goldberg, Professor of Physics
\end{singlespace}

ABSTRACT

\end{center}
\vspace{-15 pt}  % For some reason puts too much space here

Graphene is a single atomic sheet of graphite that exhibits a diverse range of unique properties.
The electrons in intrinsic graphene behave like relativistic Dirac fermions; graphene has a record high Young's modulus but extremely low bending rigidity; and suspended graphene exhibits very high thermal conductivity.
These properties are made more intriguing because with a thickness of only one a single atomic layer, graphene is both especially affected by its environment and readily manipulated.
In this dissertation the interaction between graphene and its environment as well as the exciting new physics realized by manipulating graphene's lattice are investigated.

Lattice manipulations in the form of strain cause alterations in graphene's electrical dispersion mathematically analogous to the vector potential associated with a magnetic field.
We complete the standard description of the strain-induced vector potential by explicitly including the lattice deformations and find new, leading order terms.
Additionally, a strain engineered device with large, localized, plasmonically enhanced pseudomagnetic fields is proposed to couple light to pseudomagnetic fields.

Accurate strain engineering requires a complete understanding of the interactions between a two dimensional material and its environment, particularly the adhesion and friction between graphene and its supporting substrate.
We measure the load dependent sliding friction between mono-, bi-, and trilayer graphene and the commonly used SiO$_2$ substrate by analyzing Raman spectra of circular, graphene sealed microchambers under variable external pressure.
We find that the sliding friction for trilayer graphene behaves normally, scaling with the applied load, whereas the friction for monolayer and bilayer graphene is anomalous, scaling with the inverse of the strain in the graphene.

Both strain and graphene's environment are expected to affect the quadratically dispersed out of plane acoustic phonon.
Although this phonon is believed to provide the majority of graphene's very high thermal conductivity, its contributions have never been isolated.
By measuring strain and pressure dependent thermal conductivity, we gain insight into the mechanism of graphene's thermal transport. 

\newpage