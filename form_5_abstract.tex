\begin{center}

\textbf{\thesistitle}\\

(Order No.\ \ \ \ \ \ \ \ \ \ \ \ \ )

\textbf{ALEXANDER KITT}\\
Boston University Graduate School of Arts and Sciences, 2014\\
\begin{singlespace}
	Major Professor: Bennett B. Goldberg, Professor of Physics, Professor of Electrical \\
	\qquad \qquad \qquad \quad \ \ and Computer Engineering, Professor of Biomedical Engineering
\end{singlespace}

ABSTRACT

\end{center}
\vspace{-15 pt}  % For some reason puts too much space here

Graphene, a single atomic sheet of graphite, exhibits a diverse range of unique properties.
The electrons in intrinsic graphene behave like relativistic Dirac fermions, it has a record Young's modulus but extremely low bending rigidity, and suspended graphene has record thermal conductivity.
These properties are made more intriguing because with a thickness of only one atomic layer, graphene is uncommonly affected by its environment.
This allows for graphene's atomic lattice to be manipulated and its amazing properties to be altered.
In this dissertation, we investigate the exciting new physics realized by manipulating graphene's lattice.

In an exotic coupling, strain causes alterations in graphene's electrical dispersion mathematically analogous to the vector potential associated with a magnetic field.
Here we complete the standard description of the strain-induced vector potential with the explicit inclusion of the lattice deformations.
Additionally, we propose several strain engineered devices which generate interesting pseudo magnetic fields.

Accurate strain engineering requires a complete understanding of the often bizarre interactions between a two dimensional material and its environment.
We extract the pressure dependent sliding friction between the SiO$_2$ substrate and mono-, bi-, and trilayer graphene using Raman spectroscopy of circular, graphene sealed microchambers under variable external pressure.
The sliding friction for trilayer graphene behaves normally, scaling with the applied load, whereas the friction for monolayer and bilayer graphene is anomalous, scaling with the inverse of the strain in the graphene.

Strain also modifies the phonon which is believed to provide the majority of graphene's record thermal conductivity, the out of plane acoustic phonon.
Using similar techniques we demonstrate that graphene's thermal conductivity decreases with strain, providing additional insight into the record thermal conductivity of suspended graphene.

Finally, we can also manipulate the lattice at the smallest length scale by exciting optical phonons.
It is predicted that a particular optical phonon can be used to continually push the system into an out of equilibrium state that has a transport band gap.
We propose a measurement which uses a neon seeded helium beam to excite the optical phonon and electrical transport to measure the development of the band gap.

\newpage