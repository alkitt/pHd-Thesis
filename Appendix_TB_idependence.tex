\chapter{Slowly varying approximation \label{chap:idep}}

A full, rigorous treatment of a non-uniform strain field is not compatible with the pseudo vector potential interpretation of strain developed in Chapter \ref{chap:TB}.
In a strict treatment of non-uniform strain the $i$ dependencies of Equation \ref{eq:TB:beforeSV} cannot be neglected and the transformation into Fourier space is obstructed.
Without elimination of the spatial dependence, the Hamiltonian of the full system must be painstakingly solved.
For properly engineered strain distributions, this would result in a dispersion relationship which exhibits Landau like levels.
Although rigorous, this method is more arduous and also less conceptually pleasing.
The pseudo vector potential approach provides a qualitative framework for understanding the observed effects.

In the slowly varying approximation the $i$ dependencies in Equation \ref{eq:TB:beforeSV} are eliminated.
This is as if the the strain is treated as locally uniform.
In this Appendix this approximation will be developed and the limits of validity will be discussed.

Remembering that $\bm{\nabla u}$ is $i$ dependent, the $i$ dependent terms in Equation \ref{eq:TB:beforeSV} are 
\begin{align*}
  H_i=&\sum_i \left( t_0+\delta t_{i,j} \right) e^{i(\vec{k}-\vec{k}')\cdot \vec{R}_i'}
  		e^{-i \vec{k}' \cdot \vec{\Delta}_{i,j}'} \\
  	 =&\sum_i \left( t_0+\delta t_{i,j} \right) e^{i(\vec{k}-\vec{k}')\cdot \vec{R}_i'}
  		e^{-i \vec{k}' \cdot (\bm{1}+\bm{\nabla u}) \vec{\Delta}_j} \\
  	 \simeq & e^{-i \vec{k}' \cdot \vec{\Delta}_j} \sum_i \left( t_0+\delta t_{i,j} \right)
  	 	e^{i(\vec{k}-\vec{k}')\cdot \vec{R}_i'} \left(1-i \vec{k}' \cdot \bm{\nabla u} \cdot \vec{\Delta}_j \right) \\
  	 \simeq & e^{-i \vec{k}' \cdot \vec{\Delta}_j} \bigg \{
  	 	t_0 \sum_i e^{i(\vec{k}-\vec{k}')\cdot \vec{R}_i'} 
  	 	-i t_0 \vec{k}' \cdot \sum_i \left( \bm{\nabla u} \ e^{i(\vec{k}-\vec{k}')\cdot \vec{R}_i'} \right) \cdot \vec{\Delta}_j
  	 	+\sum_i \delta t_{i,j} \  e^{i(\vec{k}-\vec{k}')\cdot \vec{R}_i'}
  	 	\bigg \} \\
  	 =& N t_0 \delta_{\vec{k},\vec{k}'}
  	 	-i t_0 \vec{k}' \cdot \widetilde{\bm{\nabla u}}_{\vec{k}-\vec{k}'} \cdot \vec{\Delta}_j + \widetilde{\delta t}_{\vec{k}-\vec{k}'} \ ,
\end{align*}
where $\widetilde{\bm{\nabla u}}_{\vec{k}-\vec{k}'}$ and $\widetilde{\delta t}_{\vec{k}-\vec{k}'}$ are the Fourier transforms of $\delta t_{i,j}$ and $\bm{\nabla u}$ respectively.
Only terms first order in products of the small quantities $\bm{\nabla u}$ and $\delta t_{i,j}$ were kept.
All of the $i$ dependence has been absorbed by the Fourier transforms.

Only specific Fourier components yield relevant $\widetilde{\bm{\nabla u}}_{\vec{k}-\vec{k}'}$ and $\widetilde{\delta t}_{\vec{k}-\vec{k}'}$ when working in the low energy regime.
The wave-vectors will again be approximated as $\vec{k}=\bm{K}+\vec{q}$ and $\vec{k}=\bm{K'}+\vec{q}$ with the additional small parameter $qa$, giving
\begin{align*}
  \widetilde{\delta t}_{\vec{k}-\vec{k}'} &=
    \sum_i \delta t_{i,j} e^{i(\bm{K^{(')}}+\vec{q}-\bm{K^{(')}}-\vec{q}') \cdot \vec{R}_i'} \\
                              &\simeq 
    \sum_i \delta t_{i,j} e^{i(\bm{K^{(')}}-\bm{K^{(')}}) \cdot \vec{R}_i'} (1+(\vec{q}-\vec{q}') \cdot \vec{R}_i') \\
                              &=
    \sum_i \delta t_{i,j} e^{i(\bm{K^{(')}}-\bm{K^{(')}}) \cdot \vec{R}_i'} \ ,
\end{align*}
where $\bm{K^{(')}}$ refers to either $\bm{K}$ or $\bm{K'}$ depending on the wave-vector.
The low energy approximation for $\widetilde{\bm{\nabla u}}_{\vec{k}-\vec{k}'}$ is identical.
Thus, to first order in the small parameters the only relevant Fourier components are for $\vec{k}-\vec{k}' \in \{0,\bm{K}-\bm{K'}, \bm{K'}-\bm{K} \}$.
The high frequency components could interestingly act to couple the $\bm{K}$ and $\bm{K}'$ points.
However, here we apply the slowly varying approximation for which we eliminate the high frequency components and limit $\vec{k}-\vec{k}' \rightarrow 0$, yielding
\begin{align*}
  \widetilde{\delta t}_{\vec{k}-\vec{k}'}     & \simeq N \delta_{\vec{k},\vec{k}'} ( <\delta t_{i,j}>) \\
  \widetilde{\bm{\nabla u}}_{\vec{k}-\vec{k}'}& \simeq N \delta_{\vec{k},\vec{k}'} ( <\bm{\nabla u} >) \ .
\end{align*}
where $<\delta t_{i,j}>=\delta t_j$ and $<\bm{\nabla u}>$ are the average value over $i$ of $\delta t_{i,j}$ and $\bm{\nabla u}$ respectively.

Thus, in the slowly varying approximation the $i$ dependence of Equation \ref{eq:TB:beforeSV} becomes
\begin{equation*}
	H_i=N \delta_{\vec{k},\vec{k}'}  \left( t_0 + \delta t_j - i t_0 \vec{k}' \cdot \bm{\nabla u} \cdot \vec{\Delta}_j \right)
\end{equation*}
which gives the same result as the simple substitutions $\delta t_{i,j} \rightarrow \delta t_j=<\delta t_{i,j}>$ and $\bm{\nabla u}_i \rightarrow <\bm{\nabla u}>$ in Equation \ref{eq:TB:beforeSV}.
This approximation required low energies, $qa<<1$, small strains, $\delta t_j << t_0$ $\bm{\nabla u}<<1$, and slowly varying strain fields, $\widetilde{\delta t}_{\bm{K}-\bm{K}'}=\widetilde{\bm{\nabla u}}_{\bm{K}-\bm{K}'}=0$.
It eliminates of the spatial dependence in the Hamiltonian of strained graphene allowing the identification of the pseudo vector potential.