\chapter{Introduction}

This thesis is a study of how the wonder material graphene responds when its lattice is manipulated.
Graphene is best described as an atomically thin bed sheet made of a single layer of graphite.
In fact, both a bed sheet and the sheets of graphene used in this thesis have a length to thickness ratio on the order of $10^6$.
Graphene was first discovered in 1969 when J.W. Mays identified graphene's low energy electron diffraction pattern \cite{May1969}. 
However, it was not until Noveselov, Geim and co-workers demonstrated both an easy method of making graphene and an easy way of seeing graphene \cite{Novoselov2004} that the field exploded in 2004.
According to Web of Science, since 2004 there have been more than 29,000 academic publications with graphene in their title \cite{WoS}.

This onslaught of publications is a result of several factors
First, graphene does not require any special facilities to fabricate.
All one needs is scotch tape, silicon wafers with thermal oxide, a microscope, and a little bit of patience.
Second, graphene is, for the most part, theoretically accessible.
The majority of its properties can be understood using a simple tight binding model \cite{CastroNeto2009}.
But most importantly, graphene has been shown to be of great scientific interest.
It has a range of exotic properties stemming from the relativistic nature of its electrons including the anomalous quantum hall effect \cite{Zhang2005} and Klein tunneling \cite{Young2009}.
It also has impressive material properties including a record Young’s modulus \cite{Lee2008}, very high thermal conductivity \cite{Faugeras2010}, and impermeability to gases \cite{Bunch2008}.

These properties are made more intriguing because with a thickness of only one atomic layer, graphene is uncommonly affected by its environment.
This allows for graphene's atomic lattice to be manipulated and its amazing properties to be altered.
The focus of this thesis is on how graphene's electrical, mechanical, and thermal properties are altered when its lattice is manipulated.
In Chapter \ref{chap:TB} the tight binding description of intrinsic graphene is reviewed.
In Chapter \ref{chap:PVP} this tight binding model is generalized to include long wavelength strains and the resulting pseudovector potentials and pseudomagnetic fields are discussed.
The generation of these strain fields require that the interactions between graphene and its underlying bulk substrate are understood.
Chapter \ref{chap:fri} describes the measurement of the anomalous macroscopic sliding friction between graphene and a SiO$_2$ substrate.
In Chapter \ref{chap:therm} it is shown that the same experimentally geometry used to study friction can be used to study the mechanism behind graphene's very high thermal conductivity.
Finally, in Chapter \ref{chap:kek} the prospects of measuring the bandgap induced by a phonon which causes small wavelength modifications of graphene's lattice are discussed.