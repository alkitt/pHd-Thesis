\chapter{Conclusion}
This thesis includes a number of new contributions to the field of manipulated grap\-hene.
For long wavelength manipulations in the form of strain, new terms in the pseudovector potential were found and new pseudomagnetic field devices were proposed.
The method of engineering the strain fields required for pseudomagnetic fields was studied in the context of the sliding friction between graphene and a SiO$_2$ substrate.
This included the discovery of the anomalous, strain dependent, macroscopic sliding friction.
Using the same experimental geometry used to study friction, we probed the mechanism behind graphene's very high thermal conductivity.
Finally, the prospects of measuring the band gap activated by the phonon induced short wavelength distortions of graphene's lattice were discussed.
These studies which ranged from the mechanical, to the optical, and to the thermal properties of graphene demonstrated how graphene's impressive properties can be extended in new and exciting directions by manipulating its lattice.