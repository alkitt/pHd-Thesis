\chapter{Hopping energies in the Kekul\'e geometry \label{chap:hopps}}

In this appendix the hopping energies between nearest neighbors in the Kekul\'e distorted lattice are calculated.
This calculation closely follows the work of Chamon \textit{et al.} \cite{Chamon2013} in which they cleverly find an expression for the nearest neighbor bond length by working in complex coordinates where vectors are expressed as $a=\vec{a} \cdot \hat{x}+i \vec{a} \cdot \hat{y}$.
They then use the change in bond length to determine the change in hopping energy using Equation \ref{eq:PVP:hopp}

The displacements generated by the iTO phonon at the $\bm{K}$ point (Equation \ref{eq:kek:displacements}) can be rewritten in complex notation as
\begin{align*}
	\vec{u}_A (\vec{r}_A,t)&=
		\left( \begin{array}{c} 
			Re[c \ e^{i\vec{r}_A \cdot \bm{K}} e^{-i \omega t}] \\
			Im[c \ e^{i\vec{r}_A \cdot \bm{K}} e^{-i \omega t}]
		\end{array} \right)
		\rightarrow u_A (\vec{r}_A,t) = c e^{i\vec{r}_A \cdot \bm{K}} e^{-i \omega t} \\
	\vec{u}_B (\vec{r}_B,t)&=
		\left( \begin{array}{c} 
			 Re[c \ e^{i\vec{r}_B \cdot \bm{K}} e^{-i \omega t}] \\
			-Im[c \ e^{i\vec{r}_B \cdot \bm{K}} e^{-i \omega t}]
		\end{array} \right)
		\rightarrow u_B (\vec{r}_B,t) = c^* e^{-i\vec{r}_B \cdot \bm{K}} e^{i \omega t} \ .
\end{align*}
The nearest neighbor vector in the Kekul\'e lattice connecting an A sub-lattice atom originally at position $\vec{r}$ and a B sub-lattice atom originally at position $\vec{r}+\vec{\delta_j}$ is 
$\vec{\delta}'_j(\vec{r})=\vec{\delta}_j+\vec{u}_B(\vec{r}+\vec{\delta}_j)-\vec{u}_A(\vec{r})$.
The magnitude of this vector is approximated as
\begin{align*}
	|\vec{\delta}'_j(\vec{r})|&=\sqrt{ \delta'_j(\vec{r}) \ \delta'_j(\vec{r})^* } \\
	&\approx \sqrt{a^2+\left\{ \delta_j^* \left(u_B(\vec{r}+\vec{\delta}_j)-u_A(\vec{r}) \right)+\text{c.c.} \right\} } \\
	&\approx a + \frac{1}{2} \frac{1}{a} \left\{ \delta_j^* \left(u_B(\vec{r}+\vec{\delta}_j)-u_A(\vec{r}) \right) + \text{c.c.}  \right\} \ .
\end{align*}
This can be simplified by using $\vec{\delta}_j=-i a z_j$ where $z_j= e^{i \bm{K} \cdot \vec{\delta}_j}$ and $z_j^2=z_j^*$.
The relative extension of the nearest neighbor vector is then
\begin{align*}
	\frac{|\vec{\delta}'_j(\vec{r})|-a}{a}=&
		\frac{1}{2} \frac{1}{a^2} (i a z_j^*) 
		\left( c^* e^{-i\vec{r} \cdot \bm{K}} z_j^* e^{i \omega t} - c \ e^{i\vec{r} \cdot \bm{K}} e^{-i \omega t} \right) \\
	&+\frac{1}{2}\frac{1}{a^2} (-ia z_j)
		\left( c \ e^{i\vec{r} \cdot \bm{K}} z_j e^{-i \omega t} - c^* e^{-i\vec{r} \cdot \bm{K}} e^{i \omega t} \right) \\
	=&i \frac{1}{2a} \left\{
		(z_j+(z_j^*)^2) c^* e^{-i\vec{r} \cdot \bm{K}} e^{i \omega t}
		-(z_j^*+z_j^2)c \ e^{i\vec{r} \cdot \bm{K}} e^{-i \omega t}
		\right\} \\
	=&-i \frac{c}{a} z_j^* e^{i\vec{r} \cdot \bm{K}} e^{-i \omega t} + \text{c.c.} \ .
\end{align*}
The use of the complex relationship between the $\delta_j$ and the $z_j$ was paramount in deriving such a simple relationship.

The term $e^{i \vec{r} \cdot \bm{K}}$ can be rewritten to have the spatial frequency which connects the inequivalent $\bm{K}$ points, $\bm{G}=\bm{K}-\bm{K'}$.
This can be shown by taking advantage of the fact that $\vec{r}$ is a lattice vector of the old lattice and that
\begin{equation*}
	\bm{K}-(\vec{b}_+-\vec{b}_-)=\bm{K}-3 \bm{K}=-2 \bm{K}=-(\bm{K}-\bm{K'})=-\bm{G} \ .
\end{equation*} 
Thus, $e^{i \vec{r}\cdot \bm{K}}=e^{i\vec{r}\cdot(\bm{K} - \vec{b}_+-\vec{b}_-)}= e^{-i \vec{r} \cdot \bm{G}}$.
In addition to making it clear that the Kekul\'e mode couples the $\bm{K}$ points, this also allows for the simplification of $\vec{r}$.
The spatial frequency $\bm{G}$ is a reciprocal lattice vector of the Kekul\'e lattice equal to $\vec{B}_+ - \vec{B}_-$.
Thus, when specifying $\vec{r}=\vec{R}_l+\vec{r}_m$ there is no need to include $\vec{R}_l$.
Instead we only specify $\vec{r}_m$, the position in the Kekul\'e unit cell of the A sub-lattice atom involved in the hopping.

The change in hopping energy resulting from the change in bond length can then be found using Equation \ref{eq:PVP:hopp}.
The resulting Kekul\'e altered hopping energies are given by
\begin{align*}
	\delta t_{m,j}&=\frac{1}{3} \Delta(t) e^{i \bm{K} \cdot \vec{\delta}_j} e^{i \bm{G} \cdot \vec{r}_m}+\text{c.c.} \nonumber \\
	& \text{with } \Delta(t)=-i 3 \beta t_0 \frac{c^*}{a} e^{i \omega t} \ .
\end{align*}
The interesting effects of these modulated hoppings are discussed in detail in Chapter \ref{chap:kek}.