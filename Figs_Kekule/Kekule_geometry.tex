% Comparing the Lattice and BZ with Kekule to without Kekule
{ % Scope so that user defined commands don't carry throughout
% Sizing constants
\newcommand{\alat}{1}
\newcommand{\amp}{.25}
\newcommand{\Klen}{2 cm}
\newcommand{\psize}{1.4 mm}
\newcommand{\sqth}{1.73205080757}

% Functions which draws the Kekule lattice
\newcommand{\kekdraw}{
	\begin{scope}

		%This scope is clipped to limit the drawn lattice to a square
		\clip (-3.9cm,-4.38cm) rectangle(3.9cm,3cm);
		
		% Cycle through the lattice points
		\foreach \ip in {-3,-2,...,3}
			\foreach \im in {-3,-2,...,3}
			{
			% Draw the unkekuled lattice
			\node at ($\ip*\sqth*\alat/2*(1,{\sqth})+\im*\sqth*\alat/2*(-1,{\sqth})+(0,+\alat/2)$) [Au] {};
			\node at ($\ip*\sqth*\alat/2*(1,{\sqth})+\im*\sqth*\alat/2*(-1,{\sqth})+(0,-\alat/2)$) [Bu] {};

			% Draw the Kekuled lattice
			\node at ($\ip*\sqth*\alat/2*(1,{\sqth})+\im*\sqth*\alat/2*(-1,{\sqth})+(0,+\alat/2)
				+\amp*({cos(120*(\ip-\im)-90)},{+sin(120*(\ip-\im)-90)} )$)            [A] {};
			\node at ($\ip*\sqth*\alat/2*(1,{\sqth})+\im*\sqth*\alat/2*(-1,{\sqth})+(0,-\alat/2)
				+\amp*({cos(120*(\ip-\im)-90)},{-sin(120*(\ip-\im)-90)} )$)            [B] {};
			}

	\end{scope}
}

\begin{tikzpicture}[scale=0.5]
	% Real space
	\begin{scope}[xshift=-5.25cm, scale=1,>=stealth,
		Bu/.style={circle,draw=blue!25,fill=blue!10,
			thick,minimum size=\psize,inner sep=0pt}, 					% Unkekuled A sublattice dots
		Au/.style={circle,draw=orange!35,fill=orange!20,
			thick,minimum size=\psize,inner sep=0pt},					% Unkekuled B sublattice dots
		B/.style={circle,draw=blue!50,fill=blue!20,
			thick,minimum size=\psize,inner sep=0pt}, 					% Kekuled A sublattice dots
		A/.style={circle,draw=orange!70,fill=orange!40,
			thick,minimum size=\psize,inner sep=0pt},					% Kekuled B sublattice dots
		nnarrow/.style={color=black,very thick, ->}]					% Arrows
		% Draw the lattice
		\kekdraw

		% Draw the primitive lattice vectors
		\draw[nnarrow] (0,-\alat*5/2) -- +( 30:3*\alat)node[anchor=south]{$\vec{A}_+$};
		\draw[nnarrow] (0,-\alat*5/2) -- +(150:3*\alat)node[anchor=south]{$\vec{A}_-$};

		% Draw the unit cell
		\draw[dashed,draw=black!75,rounded corners=.2cm,thick] (-\alat*\sqth*3/4,-\alat*15/4) rectangle (\alat*\sqth*3/4,-\alat*3/4);

	\end{scope}


	% Reciprical space
	\begin{scope}[xshift=5.25cm,yshift=-1 cm,scale=1,
		BZold/.style={color=black!75,very thick,dashed},
		BZnew/.style={color=black,very thick},
		Bar/.style={color=black,very thick,->},
		circ2/.style={radius=1.5pt}]

		% Draw the BZ
		\draw[BZold]
			(  0:\Klen) --
			( 60:\Klen) --
			(120:\Klen) -- 
			(180:\Klen) -- 
			(240:\Klen) -- 
			(300:\Klen) -- 
			(  0:\Klen);

					% Draw the BZ
		\draw[BZnew]
			( 30:\Klen/\sqth) --
			( 90:\Klen/\sqth) --
			(150:\Klen/\sqth) -- 
			(210:\Klen/\sqth) -- 
			(270:\Klen/\sqth) -- 
			(330:\Klen/\sqth) -- 
			( 30:\Klen/\sqth);

		\draw[Bar] (0,0) -- ( 60:\Klen) node[anchor=south west]{$\vec{B}_+$};
		\draw[Bar] (0,0) -- (120:\Klen) node[anchor=south east]{$\vec{B}_-$};

		% Label the high symmetry points
		\draw[fill=black] (0,0) circle[circ2] node[anchor=west]{$\Gamma$};
		\draw[fill=black] (\Klen/2,0) circle[circ2] node[anchor=west]{$M'$};
	\end{scope}
	\node at (-9.5cm,3.5cm) {\textbf{(a)}};
	\node at ( 2.5cm,3.5cm) {\textbf{(b)}};
\end{tikzpicture}
}